% SuChengChoice.tex
% 子文件:选择题部分
% 请确保主文件已加载 msp-exam 文档类 (或包含 ctex, amsmath, enumitem 等宏包)

\section{基础概念选择题}

% --- 第 1 题 ---
\begin{problem}[功率谱密度的性质]
    下列哪些函数是实随机信号的功率谱密度的正确表达式( \quad )
    \begin{enumerate}[label=(\Alph*), itemsep=1em]
        \item $\displaystyle \frac{w^2}{w^6 + 3w^2 + 3}$
        \item $\displaystyle \frac{w^2}{jw^6 + w^2 + 1}$
        \item $\displaystyle \frac{w^2}{w^3 - 1} - \delta(w)$
        \item $\displaystyle e^{-(w-1)^2}$
    \end{enumerate}
    
    \begin{solution}
        \textbf{正确答案:(A)}
       
        \textbf{分析:} 实随机信号的功率谱密度 $S_x(w)$ 必须满足以下三个性质:
        \begin{enumerate}
            \item \textbf{实值性}:功率谱必须是实函数。选项 (B) 分母含有虚数单位 $j$,故排除。
            \item \textbf{非负性}:$S_x(w) \ge 0$。选项 (C) 中 $\delta(w)$ 前为负号,且分母 $w^3-1$ 在 $w<1$ 时可能为负,故排除。
            \item \textbf{偶对称性}:$S_x(w) = S_x(-w)$。
            \begin{itemize}
                \item 选项 (C) 分母含 $w^3$ 为奇函数项,不满足偶对称。
                \item 选项 (D) 为中心在 $w=1$ 的高斯函数,不关于原点偶对称,故排除。
            \end{itemize}
        \end{enumerate}
        检查选项 (A):
        $$ S(w) = \frac{w^2}{w^6 + 3w^2 + 3} $$
        分子 $w^2$ 是偶函数且非负;分母 $w^6 + 3w^2 + 3$ 是偶函数且恒大于0(各项均为非负项且常数项为3)。因此整个式子是实值、非负且偶对称的。
    \end{solution}
\end{problem}

% --- 第 2 题 ---
\begin{problem}[白噪声的定义]
    下面属于白噪声的信号是( \quad )
    \begin{enumerate}[label=(\Alph*), itemsep=0.5em]
        \item 带宽大于系统的带宽,并且在系统的带宽内该信号的频谱近似均匀分布
        \item 功率谱在整个频域内是均匀分布的信号
        \item 序列 $x(n)$ 的变量是两两不相关的
        \item 符合正态分布的随机信号
    \end{enumerate}

    \begin{solution}
        \textbf{正确答案:(ABC)}
        
        \textbf{分析:} 
        \begin{itemize}
            \item (A) 描述的是工程上实用的“带限白噪声”或针对特定系统的“准白噪声”。
            \item (B) 是白噪声的严格定义,理想白噪声在整个频域内功率谱密度均匀分布。
            \item (C) 描述的是离散白噪声的时域特性(自相关函数为冲激函数,即样本间不相关)。
            \item (D) 描述的是高斯噪声(幅度分布特性),高斯噪声不一定是白噪声(频谱特性),反之亦然。
        \end{itemize}
    \end{solution}
\end{problem}

% --- 第 3 题 ---
\begin{problem}[FIR与IIR滤波器对比]
    FIR 滤波器与 IIR 滤波器相比较,下列说法中正确的是( \quad )
    \begin{enumerate}[label=(\Alph*), itemsep=0.5em]
        \item FIR 相位可以做到严格线性
        \item FIR 主要是非递归结构
        \item 相同性能下 FIR 阶次高
        \item 以上说法都不对
    \end{enumerate}

    \begin{solution}
        \textbf{正确答案:(ABC)}
        
        \textbf{分析:} 
        \begin{itemize}
            \item \textbf{(A) 正确}:FIR 滤波器只要满足系数对称或反对称($h[n] = \pm h[N-1-n]$),即可具有严格的线性相位特性。这是 FIR 相比 IIR 最大的优势。
            \item \textbf{(B) 正确}:FIR(有限长单位冲激响应)通常采用非递归结构(如横截型结构),系统函数只有零点,没有极点(极点在原点),保证了绝对稳定性。
            \item \textbf{(C) 正确}:为了达到与 IIR 相同的幅频特性(如下降陡度),FIR 通常需要比 IIR 高得多的阶数(Tap数),因为 IIR 利用反馈引入了极点,能更高效地逼近理想特性。
        \end{itemize}
    \end{solution}
\end{problem}

% --- 第 4 题 ---
\begin{problem}[时域与频域的关系]
    下述对时域与频域之间关系描述正确的是( \quad )
    \begin{enumerate}[label=(\Alph*), itemsep=0.5em]
        \item 信号的相关性越强,它的功率谱就越窄
        \item 信号的功率谱越宽,说明它的相关性越弱
        \item 信号在时域分布越宽,则在频域分布就越窄
        \item 信号在时间上压缩 $N$ 倍,则其频谱就在频域打展 $N$ 倍
    \end{enumerate}

    \begin{solution}
        \textbf{正确答案:(ABCD)}
        
        \textbf{分析:} 
        \begin{itemize}
            \item \textbf{(A)(B) 正确}:相关性与功率谱(或频谱)是一对傅里叶变换关系。时域变化越缓慢(相关性越强),频域分量越集中于低频(功率谱越窄);反之,变化越剧烈(不相关/白噪声),功率谱越宽。
            \item \textbf{(C) 正确}:根据不确定性原理(时宽频宽积为常数),时域的扩展对应频域的压缩,时域的宽分布对应频域的窄分布。
            \item \textbf{(D) 正确}:根据傅里叶变换的尺度变换性质(Scaling Property),$x(at) \leftrightarrow \frac{1}{|a|}X(\frac{w}{a})$。若 $a=N$(压缩),则频域变量变为 $w/N$,即频域宽度扩展 $N$ 倍。
        \end{itemize}
    \end{solution}
\end{problem}

% --- 第 5 题 ---
\begin{problem}[周期图法谱估计的性质]
    周期图法谱估计的方差:( \quad )
    \begin{enumerate}[label=(\Alph*), itemsep=0.5em]
        \item 大,但随着数据长度增大而减小
        \item 小,但随着数据长度增大而增大
        \item 大,且不随着数据长度增大而减小
        \item 小,且随着数据长度增大而减小
    \end{enumerate}

    \begin{solution}
        \textbf{正确答案:(C)}
        
        \textbf{分析:} 
        周期图法(Periodogram)是功率谱密度的渐近无偏估计,但它是非一致(inconsistent)估计。
        \begin{itemize}
            \item 随着数据长度 $N \to \infty$,其方差并不趋于 0。
            \item 事实上,周期图谱估计的方差与真实功率谱值的平方成正比,即 $\text{Var}[\hat{P}(w)] \approx P^2(w)$。
            \item 无论 $N$ 多大,方差都保持较大水平,因此需要采用平均周期图法(Bartlett法)或平滑周期图法(Welch法)来减小方差。
        \end{itemize}
    \end{solution}
\end{problem}

% --- 第 6 题 ---
\begin{problem}[自适应滤波器的性能]
    自适应滤波器的性能最终取决于( \quad )
    \begin{enumerate}[label=(\Alph*), itemsep=0.5em]
        \item 采用横向结构还是采用格型结构
        \item 其数字滤波器是递归的还是非递归的
        \item 所采用的自适应算法
        \item 输入序列与参考序列之间的相关程度
    \end{enumerate}

    \begin{solution}
        \textbf{正确答案:(D)}
        
        \textbf{分析:} 
        \begin{itemize}
            \item 选项 (A)、(B)、(C) 主要影响滤波器的收敛速度、计算复杂度以及数值稳定性。例如,RLS 算法比 LMS 算法收敛快,IIR 结构比 FIR 结构更节省阶数。
            \item 选项 (D) 决定了滤波器的最终效果(最小均方误差的下限)。自适应滤波的本质是从输入信号中提取出与参考信号相关的部分。如果输入信号与参考信号完全不相关,无论用什么结构或算法,滤波器都无法有效工作。维纳解(Wiener Solution)表明最优权值取决于互相关向量 $\mathbf{p}$ 和自相关矩阵 $\mathbf{R}$。
        \end{itemize}
    \end{solution}
\end{problem}

% --- 第 7 题 ---
\begin{problem}[因果稳定系统与零极点]
    一个因果稳定的可逆系统,其系统函数 $H(z)$ 的零、极点分布情况:( \quad )
    \begin{enumerate}[label=(\Alph*), itemsep=0.5em]
        \item 零点在 $z$ 平面的单位圆的内部,极点在 $z$ 平面的单位圆的外部
        \item 零、极点全部在 $z$ 平面的单位圆的内部
        \item 零点在 $z$ 平面的单位圆的外部,极点在 $z$ 平面的单位圆的内部
        \item 零、极点全部在 $z$ 平面的单位圆的外部
    \end{enumerate}

    \begin{solution}
        \textbf{正确答案:(B)}
        
        \textbf{分析:} 
        \begin{itemize}
            \item \textbf{因果稳定性}:系统因果且稳定,要求其系统函数 $H(z)$ 的所有极点必须位于 $z$ 平面单位圆的\textbf{内部}。
            \item \textbf{可逆性(最小相位)}:题目中的“可逆系统”通常指逆系统也是因果稳定的(即最小相位系统)。逆系统的系统函数为 $H^{-1}(z) = 1/H(z)$,其极点即为原系统 $H(z)$ 的零点。为了保证逆系统也是因果稳定的,原系统的零点也必须位于单位圆的\textbf{内部}。
            \item 综上所述,该系统的零点和极点都必须全部在单位圆内部。
        \end{itemize}
    \end{solution}
\end{problem}

% --- 第 8 题 ---
\begin{problem}[随机信号通过线性系统]
    对于随机信号通过线性系统,下面正确的概念是( \quad )
    \begin{enumerate}[label=(\Alph*), itemsep=0.5em]
        \item 输入输出互相关序列就是系统的单位脉冲响应序列
        \item 输出自功率谱等于输入自功率谱乘上系统的能量谱
        \item 输入输出的互功率谱等于输入的自功率谱与系统频率响应的乘积
        \item 输入输出互相关序列等于输入自相关序列乘上系统的单位脉冲响应序列
    \end{enumerate}

    \begin{solution}
        \textbf{正确答案:(BC)}
        
        \textbf{分析:} 
        \begin{itemize}
            \item \textbf{(A) 错误}:输入输出互相关函数 $R_{xy}(m) = R_{xx}(m) * h(m)$。只有当输入是白噪声($R_{xx}(m)=\delta(m)$)时,互相关函数才等于系统的单位脉冲响应 $h(m)$。
            \item \textbf{(B) 正确}:对于线性时不变系统,输出功率谱密度 $S_{yy}(w)$ 与输入功率谱密度 $S_{xx}(w)$ 的关系为 $S_{yy}(w) = S_{xx}(w) |H(w)|^2$。其中 $|H(w)|^2$ 常被称为系统的能量谱(或功率传输函数)。
            \item \textbf{(C) 正确}:输入输出互功率谱 $S_{xy}(w)$ 与输入自功率谱的关系为 $S_{xy}(w) = S_{xx}(w) H(w)$(注意:此处假设互相关定义为 $R_{xy}(\tau) = E[x(t)y(t+\tau)]$,若定义不同可能涉及共轭)。
            \item \textbf{(D) 错误}:时域中,输入输出互相关序列是输入自相关序列与系统单位脉冲响应的\textbf{卷积},而不是乘积。
        \end{itemize}
    \end{solution}
\end{problem}

% --- 第 9 题 ---
\begin{problem}[信号的频域分辨率]
    对于信号的频域分辨率,下面正确的概念是( \quad )
    \begin{enumerate}[label=(\Alph*), itemsep=0.5em]
        \item 采用新的信号处理算法可以提高信号的频域分辨率
        \item 增大信号时域的实际长度可以提高信号的频域分辨率
        \item 信号的频域分辨率与时域截断时所采用的窗函数有关
        \item 信号的频域分辨率与作 DFT 时补零的长度有关
    \end{enumerate}

    \begin{solution}
        \textbf{正确答案:(ABC)}
        
        \textbf{分析:} 
        \begin{itemize}
            \item \textbf{(A) 正确}:传统的非参数谱估计(如周期图法)分辨率受限于数据长度(瑞利限),但现代参数谱估计算法(如 AR 模型、MUSIC、ESPRIT 等)可以突破这一限制,显著提高分辨率。
            \item \textbf{(B) 正确}:物理分辨率主要取决于信号的观测时长 $T$(或采样点数 $N$),分辨率 $\Delta f \approx 1/T$。增加数据长度是提高物理分辨率最直接的方法。
            \item \textbf{(C) 正确}:加窗会使谱峰变宽(主瓣加宽),导致分辨率下降。不同窗函数的主瓣宽度不同,因此分辨率与窗函数有关(如矩形窗主瓣最窄,分辨率最高;黑曼窗主瓣宽,分辨率低)。
            \item \textbf{(D) 错误}:补零(Zero-padding)只是对频谱进行插值,增加了频谱采样的密度(栅栏效应减小),使谱线看起来更平滑,但并不能区分两个原本无法分辨的紧邻频率分量,即不能提高物理分辨率。
        \end{itemize}
    \end{solution}
\end{problem}

% --- 第 9 题 ---
