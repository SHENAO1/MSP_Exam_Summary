% SuChengChoice.tex
% 选择题部分
% 配合 msp-exam.cls v2.1 使用,会自动生成“题干页 -> 解析页”的双页模式

\section{基础概念选择题}

% --- 第 1 题 ---
\begin{choiceproblem}[功率谱密度的性质]
    下列哪些函数是实随机信号的功率谱密度的正确表达式( \quad )
    \begin{enumerate}[label=(\Alph*), itemsep=1em]
        \item $\displaystyle \frac{w^2}{w^6 + 3w^2 + 3}$
        \item $\displaystyle \frac{w^2}{jw^6 + w^2 + 1}$
        \item $\displaystyle \frac{w^2}{w^3 - 1} - \delta(w)$
        \item $\displaystyle e^{-(w-1)^2}$
    \end{enumerate}
    
    \begin{solution}
        \textbf{正确答案:(A)}
        
        \textbf{分析:} 实随机信号的功率谱密度 $S_x(w)$ 必须满足以下三个性质:
        \begin{enumerate}
            \item \textbf{实值性}:功率谱必须是实函数。选项 (B) 分母含有虚数单位 $j$,故排除。
            \item \textbf{非负性}:$S_x(w) \ge 0$。选项 (C) 中 $\delta(w)$ 前为负号,且分母 $w^3-1$ 在 $w<1$ 时可能为负,故排除。
            \item \textbf{偶对称性}:$S_x(w) = S_x(-w)$。
            \begin{itemize}
                \item 选项 (C) 分母含 $w^3$ 为奇函数项,不满足偶对称。
                \item 选项 (D) 为中心在 $w=1$ 的高斯函数,不关于原点偶对称,故排除。
            \end{itemize}
        \end{enumerate}
        检查选项 (A):
        $$ S(w) = \frac{w^2}{w^6 + 3w^2 + 3} $$
        分子 $w^2$ 是偶函数且非负;分母 $w^6 + 3w^2 + 3$ 是偶函数且恒大于0。因此整个式子是实值、非负且偶对称的。
    \end{solution}
\end{choiceproblem}

% --- 第 2 题 ---
\begin{choiceproblem}[白噪声的定义]
    下面属于白噪声的信号是( \quad )
    \begin{enumerate}[label=(\Alph*), itemsep=0.5em]
        \item 带宽大于系统的带宽,并且在系统的带宽内该信号的频谱近似均匀分布
        \item 功率谱在整个频域内是均匀分布的信号
        \item 序列 $x(n)$ 的变量是两两不相关的
        \item 符合正态分布的随机信号
    \end{enumerate}

    \begin{solution}
        \textbf{正确答案:(ABC)}
        
        \textbf{分析:} 
        \begin{itemize}
            \item (A) 描述的是工程上实用的“带限白噪声”或针对特定系统的“准白噪声”。
            \item (B) 是白噪声的严格定义,理想白噪声在整个频域内功率谱密度均匀分布。
            \item (C) 描述的是离散白噪声的时域特性(自相关函数为冲激函数,即样本间不相关)。
            \item (D) 描述的是高斯噪声(幅度分布特性),高斯噪声不一定是白噪声(频谱特性),反之亦然。
        \end{itemize}
    \end{solution}
\end{choiceproblem}

% --- 第 3 题 ---
\begin{choiceproblem}[FIR与IIR滤波器对比]
    FIR 滤波器与 IIR 滤波器相比较,下列说法中正确的是( \quad )
    \begin{enumerate}[label=(\Alph*), itemsep=0.5em]
        \item FIR 相位可以做到严格线性
        \item FIR 主要是非递归结构
        \item 相同性能下 FIR 阶次高
        \item 以上说法都不对
    \end{enumerate}

    \begin{solution}
        \textbf{正确答案:(ABC)}
        
        \textbf{分析:} 
        \begin{itemize}
            \item \textbf{(A) 正确}:FIR 滤波器只要满足系数对称或反对称,即可具有严格的线性相位特性。
            \item \textbf{(B) 正确}:FIR 通常采用非递归结构,系统函数只有零点,保证了绝对稳定性。
            \item \textbf{(C) 正确}:为了达到与 IIR 相同的幅频特性,FIR 通常需要比 IIR 高得多的阶数。
        \end{itemize}
    \end{solution}
\end{choiceproblem}

% --- 第 4 题 ---
\begin{choiceproblem}[时域与频域的关系]
    下述对时域与频域之间关系描述正确的是( \quad )
    \begin{enumerate}[label=(\Alph*), itemsep=0.5em]
        \item 信号的相关性越强,它的功率谱就越窄
        \item 信号的功率谱越宽,说明它的相关性越弱
        \item 信号在时域分布越宽,则在频域分布就越窄
        \item 信号在时间上压缩 $N$ 倍,则其频谱就在频域打展 $N$ 倍
    \end{enumerate}

    \begin{solution}
        \textbf{正确答案:(ABCD)}
        
        \textbf{分析:} 
        \begin{itemize}
            \item \textbf{(A)(B) 正确}:相关性与功率谱是一对傅里叶变换关系。时域变化缓慢(相关性强)对应频域窄;变化剧烈(不相关)对应频域宽。
            \item \textbf{(C) 正确}:根据不确定性原理(时宽频宽积为常数)。
            \item \textbf{(D) 正确}:根据傅里叶变换的尺度变换性质。
        \end{itemize}
    \end{solution}
\end{choiceproblem}

% --- 第 5 题 ---
\begin{choiceproblem}[周期图法谱估计的性质]
    周期图法谱估计的方差:( \quad )
    \begin{enumerate}[label=(\Alph*), itemsep=0.5em]
        \item 大,但随着数据长度增大而减小
        \item 小,但随着数据长度增大而增大
        \item 大,且不随着数据长度增大而减小
        \item 小,且随着数据长度增大而减小
    \end{enumerate}

    \begin{solution}
        \textbf{正确答案:(C)}
        
        \textbf{分析:} 
        周期图法是功率谱密度的非一致估计。
        \begin{itemize}
            \item 随着数据长度 $N \to \infty$,其方差并不趋于 0。
            \item $\text{Var}[\hat{P}(w)] \approx P^2(w)$,无论 $N$ 多大,方差都保持较大水平。
        \end{itemize}
    \end{solution}
\end{choiceproblem}

% --- 第 6 题 ---
\begin{choiceproblem}[自适应滤波器的性能]
    自适应滤波器的性能最终取决于( \quad )
    \begin{enumerate}[label=(\Alph*), itemsep=0.5em]
        \item 采用横向结构还是采用格型结构
        \item 其数字滤波器是递归的还是非递归的
        \item 所采用的自适应算法
        \item 输入序列与参考序列之间的相关程度
    \end{enumerate}

    \begin{solution}
        \textbf{正确答案:(D)}
        
        \textbf{分析:} 
        自适应滤波的本质是从输入信号中提取出与参考信号相关的部分。如果输入信号与参考信号完全不相关,无论用什么结构或算法,滤波器都无法有效工作。
    \end{solution}
\end{choiceproblem}

% --- 第 7 题 ---
\begin{choiceproblem}[因果稳定系统与零极点]
    一个因果稳定的可逆系统,其系统函数 $H(z)$ 的零、极点分布情况:( \quad )
    \begin{enumerate}[label=(\Alph*), itemsep=0.5em]
        \item 零点在 $z$ 平面的单位圆的内部,极点在 $z$ 平面的单位圆的外部
        \item 零、极点全部在 $z$ 平面的单位圆的内部
        \item 零点在 $z$ 平面的单位圆的外部,极点在 $z$ 平面的单位圆的内部
        \item 零、极点全部在 $z$ 平面的单位圆的外部
    \end{enumerate}

    \begin{solution}
        \textbf{正确答案:(B)}
        
        \textbf{分析:} 
        \begin{itemize}
            \item \textbf{因果稳定性}:极点必须位于单位圆\textbf{内部}。
            \item \textbf{可逆性(最小相位)}:逆系统也是因果稳定的,要求原系统的零点也必须位于单位圆的\textbf{内部}。
        \end{itemize}
    \end{solution}
\end{choiceproblem}

% --- 第 8 题 ---
\begin{choiceproblem}[随机信号通过线性系统]
    对于随机信号通过线性系统,下面正确的概念是( \quad )
    \begin{enumerate}[label=(\Alph*), itemsep=0.5em]
        \item 输入输出互相关序列就是系统的单位脉冲响应序列
        \item 输出自功率谱等于输入自功率谱乘上系统的能量谱
        \item 输入输出的互功率谱等于输入的自功率谱与系统频率响应的乘积
        \item 输入输出互相关序列等于输入自相关序列乘上系统的单位脉冲响应序列
    \end{enumerate}

    \begin{solution}
        \textbf{正确答案:(BC)}
        
        \textbf{分析:} 
        \begin{itemize}
            \item (A) 仅当输入为白噪声时成立。
            \item (B) $S_{yy}(w) = S_{xx}(w) |H(w)|^2$,正确。
            \item (C) $S_{xy}(w) = S_{xx}(w) H(w)$,正确。
            \item (D) 时域中应为\textbf{卷积},而非乘积。
        \end{itemize}
    \end{solution}
\end{choiceproblem}

% --- 第 9 题 ---
\begin{choiceproblem}[信号的频域分辨率]
    对于信号的频域分辨率,下面正确的概念是( \quad )
    \begin{enumerate}[label=(\Alph*), itemsep=0.5em]
        \item 采用新的信号处理算法可以提高信号的频域分辨率
        \item 增大信号时域的实际长度可以提高信号的频域分辨率
        \item 信号的频域分辨率与时域截断时所采用的窗函数有关
        \item 信号的频域分辨率与作 DFT 时补零的长度有关
    \end{enumerate}

    \begin{solution}
        \textbf{正确答案:(ABC)}
        
        \textbf{分析:} 
        \begin{itemize}
            \item (A) 现代谱估计(如 MUSIC)可以突破瑞利限。
            \item (B) 物理分辨率 $\Delta f \approx 1/T$,增加时长 $T$ 可提高分辨率。
            \item (C) 窗函数主瓣越宽,分辨率越低。
            \item (D) 补零只能减小栅栏效应,不能提高物理分辨率。
        \end{itemize}
    \end{solution}
\end{choiceproblem}

% --- 第 10 题 ---
\begin{choiceproblem}[滤波器结构与算法特性]
    下面说法正确的是( \quad )
    \begin{enumerate}[label=(\Alph*), itemsep=0.5em]
        \item IIR 滤波器级联结构比直接型结构对系数变换的敏感度小,受有限字长的影响低
        \item 在线性相位 FIR 滤波器结构中,利用系数的对称性,相同系数共用乘法器,可以节省近一半的乘法器
        \item 白噪声是相关性最差的随机序列,谐波是相关性最强的随机序列
        \item 自适应滤波算法中,RLS 算法与 LMS 算法相比,RLS 算法的收敛速率快,而且失调量小,但计算复杂度高
    \end{enumerate}

    \begin{solution}
        \textbf{正确答案:(ABCD)}
    \end{solution}
\end{choiceproblem}

% --- 第 11 题 ---
\begin{choiceproblem}[离散系统的性质]
    关于离散系统,下面叙述正确的是:( \quad )
    \begin{enumerate}[label=(\Alph*), itemsep=0.5em]
        \item 系统 $y(n) = 2x(n) + 1$ 是线性系统
        \item $y(n) = x(2n)$ 是时不变系统
        \item 系统 $y(n) = [x(n+1) + x(n) + x(n-1)]/3$ 是因果系统
        \item 如果 FIR 系统的单位脉冲响应 $h(n)$ 是实数,且是偶对称的,即满足:$h(n)=h(N-n-1)$,则该系统一定具有线性相位特性
    \end{enumerate}

    \begin{solution}
        \textbf{正确答案:(D)}
        
        \textbf{分析:} 
        \begin{itemize}
            \item (A) 存在常数项 $+1$,不满足零输入零输出。
            \item (B) 抽取系统是时变系统。
            \item (C) 依赖未来输入 $x(n+1)$,非因果。
            \item (D) 满足对称性的 FIR 具有线性相位。
        \end{itemize}
    \end{solution}
\end{choiceproblem}

% --- 第 12 题 ---
\begin{choiceproblem}[MATLAB信号处理函数的使用]
    关于 MATLAB 函数的使用,下面叙述正确的是:( \quad )
    \begin{enumerate}[label=(\Alph*), itemsep=0.5em]
        \item 计算系统的频率特性可以使用函数 \texttt{freqz(b, a)},其中 $b$ 是系统函数 $H(z)$ 分母多项式的系数,$a$ 是系统函数分子多项式的系数
        \item 计算信号的频谱可以使用函数 \texttt{y=fft(x)},如果 $x$ 是 $N$ 点的信号,则 $y$ 是 $x$ 频谱在 $0 \sim 2\pi$ 的 $N$ 等分
        \item 产生正态分布的白噪声使用函数 \texttt{rand}
        \item 设计 FIR 滤波器可以使用函数 \texttt{h = fir1(N, w)},表示使用默认的窗函数设计一个 $N$ 阶,截止频率为 $w$ 的数字低通滤波器
    \end{enumerate}

    \begin{solution}
        \textbf{正确答案:(BD)}
        
        \textbf{分析:} 
        \begin{itemize}
            \item (A) 颠倒了,$b$ 是分子,$a$ 是分母。
            \item (C) \texttt{rand} 是均匀分布,正态分布应为 \texttt{randn}。
        \end{itemize}
    \end{solution}
\end{choiceproblem}

% --- 第 13 题 ---
\begin{choiceproblem}[FIR与IIR滤波器的对比]
    FIR 滤波器与 IIR 滤波器相比较,下列说法中正确的是:( \quad )
    \begin{enumerate}[label=(\Alph*), itemsep=0.5em]
        \item IIR 滤波器无法得到严格的线性相位
        \item FIR 滤波器一定是线性相位的
        \item FIR 滤波器的相位可以做到严格线性
        \item 相同性能下 FIR 滤波器的阶次高
    \end{enumerate}

    \begin{solution}
        \textbf{正确答案:(ACD)}
        
        \textbf{分析:} 
        (B) 只有满足对称条件的 FIR 才是线性相位的。
    \end{solution}
\end{choiceproblem}

% --- 第 14 题 ---
\begin{choiceproblem}[LMS算法的特性]
    自适应滤波器采用 LMS 算法迭代,正确的说法是( \quad )
    \begin{enumerate}[label=(\Alph*), itemsep=0.5em]
        \item 达到稳态后,均方误差恒定在最小值
        \item 迭代过程中,输出误差的平方单调向碗底运动
        \item 信号越强收敛越慢
        \item 失调量与步长成正比
    \end{enumerate}

    \begin{solution}
        \textbf{正确答案:(D)}
        
        \textbf{分析:} 
        \begin{itemize}
            \item (A) 存在过剩均方误差,会在最小值附近波动。
            \item (B) 梯度估计含噪声,轨迹是曲折的。
            \item (C) 信号越强,收敛越快。
            \item (D) 失调量正比于步长 $\mu$。
        \end{itemize}
    \end{solution}
\end{choiceproblem}