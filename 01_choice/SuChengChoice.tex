% SuChengChoice.tex
% 子文件:选择题部分
% 请确保主文件已加载 msp-exam 文档类

\section{基础概念选择题}

% --- 第 1 题 ---
\begin{problem}[功率谱密度的性质]
    下列哪些函数是实随机信号的功率谱密度的正确表达式( \quad )
    \begin{enumerate}[label=(\Alph*), itemsep=1em]
        \item $\displaystyle \frac{w^2}{w^6 + 3w^2 + 3}$
        \item $\displaystyle \frac{w^2}{jw^6 + w^2 + 1}$
        \item $\displaystyle \frac{w^2}{w^3 - 1} - \delta(w)$
        \item $\displaystyle e^{-(w-1)^2}$
    \end{enumerate}
    
    \begin{solution}
        \textbf{正确答案:(A)}
        
        \textbf{分析:} 实随机信号的功率谱密度 $S_x(w)$ 必须满足以下三个性质:
        \begin{enumerate}
            \item \textbf{实值性}:功率谱必须是实函数。选项 (B) 分母含有虚数单位 $j$,故排除。
            \item \textbf{非负性}:$S_x(w) \ge 0$。选项 (C) 中 $\delta(w)$ 前为负号,且分母 $w^3-1$ 在 $w<1$ 时可能为负,故排除。
            \item \textbf{偶对称性}:$S_x(w) = S_x(-w)$。
            \begin{itemize}
                \item 选项 (C) 分母含 $w^3$ 为奇函数项,不满足偶对称。
                \item 选项 (D) 为中心在 $w=1$ 的高斯函数,不关于原点偶对称,故排除。
            \end{itemize}
        \end{enumerate}
        检查选项 (A):
        $$ S(w) = \frac{w^2}{w^6 + 3w^2 + 3} $$
        分子 $w^2$ 是偶函数且非负;分母 $w^6 + 3w^2 + 3$ 是偶函数且恒大于0(各项均为非负项且常数项为3)。因此整个式子是实值、非负且偶对称的。
    \end{solution}
\end{problem}

% --- 第 2 题 ---
\begin{problem}[白噪声的定义]
    下面属于白噪声的信号是( \quad )
    \begin{enumerate}[label=(\Alph*), itemsep=0.5em]
        \item 带宽大于系统的带宽,并且在系统的带宽内该信号的频谱近似均匀分布
        \item 功率谱在整个频域内是均匀分布的信号
        \item 序列 $x(n)$ 的变量是两两不相关的
        \item 符合正态分布的随机信号
    \end{enumerate}

    \begin{solution}
        \textbf{正确答案:(B)}
        
        \textbf{分析:} 
        \begin{itemize}
            \item \textbf{(B)} 是白噪声严格的频域定义:功率谱密度在整个频率轴 $(-\infty, +\infty)$ 上为常数(均匀分布)。
            \item (A) 描述的是工程上实用的“带限白噪声”或针对特定系统的“准白噪声”,不是严格的定义。
            \item (C) 描述的是离散白噪声的时域特性(自相关函数为冲激函数),虽然正确,但通常白噪声的首要定义基于“白”(即全频谱能量均等)。
            \item (D) 描述的是高斯噪声(幅度分布特性),高斯噪声不一定是白噪声(频谱特性)。
        \end{itemize}
    \end{solution}
\end{problem}

% --- 第 3 题 ---
\begin{problem}[FIR与IIR滤波器对比]
    FIR 滤波器与 IIR 滤波器相比较,下列说法中正确的是( \quad )
    \begin{enumerate}[label=(\Alph*), itemsep=0.5em]
        \item FIR 相位可以做到严格线性
        \item FIR 主要是非递归结构
        \item 相同性能下 FIR 阶次高
        \item 以上说法都对
    \end{enumerate}

    \begin{solution}
        \textbf{正确答案:(D)}
        
        \textbf{分析:} 
        \begin{itemize}
            \item \textbf{(A) 正确}:FIR 滤波器只要满足系数对称或反对称($h[n] = \pm h[N-1-n]$),即可具有严格的线性相位特性。这是 FIR 相比 IIR 最大的优势。
            \item \textbf{(B) 正确}:FIR(有限长单位冲激响应)通常采用非递归结构(如横截型结构),系统函数只有零点,没有极点(极点在原点),保证了绝对稳定性。
            \item \textbf{(C) 正确}:为了达到与 IIR 相同的幅频特性(如下降陡度),FIR 通常需要比 IIR 高得多的阶数(Tap数),因为 IIR 利用反馈引入了极点,能更高效地逼近理想特性。
        \end{itemize}
        综上所述,(A)、(B)、(C) 均正确。
    \end{solution}
\end{problem}